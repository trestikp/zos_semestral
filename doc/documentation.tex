\documentclass[12pt]{report}


% Zařadit literaturu do obsahu
\usepackage[nottoc,notlot,notlof]{tocbibind}

% Umožňuje vkládání obrázků
\usepackage[pdftex]{graphicx}

% Odkazy v PDF jsou aktivní; navíc se automaticky vkládá
% balíček 'url', který umožňuje např. dělení slov
% uvnitř URL
\usepackage[pdftex]{hyperref}
\hypersetup{colorlinks=true,
  unicode=true,
  linkcolor=black,
  citecolor=black,
  urlcolor=black,
  bookmarksopen=true}

% Při používání citačního stylu csplainnatkiv
% (odvozen z csplainnat, http://repo.or.cz/w/csplainnat.git)
% lze snadno modifikovat vzhled citací v textu
\usepackage[numbers,sort&compress]{natbib}


% me package
\usepackage{float}
\usepackage{subfig}
\usepackage{siunitx}
\usepackage[czech]{babel}
\usepackage{tabularx}

%%%%%%%%%%%%%%%%%%%%%%%%%%%%%%%%%%%%%%%%%%%%%%%%%%%%%%%%%%
%
% VLASTNÍ TEXT PRÁCE
%
%%%%%%%%%%%%%%%%%%%%%%%%%%%%%%%%%%%%%%%%%%%%%%%%%%%%%%%%%%

\makeatletter
\def\@makechapterhead#1{%
  {\parindent \z@ \raggedright \normalfont
    \Huge\bfseries
    \ifnum \c@secnumdepth >\m@ne
        \hangindent=1.5em
        \noindent\hbox to 1.5em{\thechapter\hfil}%
    \fi%
    #1\par\nobreak
    \vskip 40\p@
  }}
\def\@makeschapterhead#1{%
  {\parindent \z@ \raggedright \normalfont
    \interlinepenalty\@m
    \huge \bfseries  #1\par\nobreak
    \vskip 40\p@
  }}
\makeatother


\begin{document}
%

\begin{titlepage}
\centerline{\includegraphics[width=10cm]{img/logo.jpg}}
\begin{center}
\vspace{30px}
{\huge
\textbf{Základy operačních systémů}\\
\vspace{1cm}
}
{\large
\textbf{KIV/ZOS}\\
\vspace{1cm}
}
\vspace{1cm}
{\large
Pavel Třeštík\\
}
{\normalsize
A17B0380P
}
\end{center}
\vspace{\fill}
\hfill
\begin{minipage}[t]{7cm}
\flushright
\today
\end{minipage}
\end{titlepage}


\tableofcontents



\chapter{Úvod}
Cílem této práce je vytvořit zjednodušný souborový systém (anglicky file system, dále jen (pseudo) FS) založený
na reálném FS i-uzlů (anglicky i-node, dále jen inode). Dále nad tímto systémem implementovat některé
základní linuxové příkazy jako například "cd, ls" a další. Tyto příkazy jsou zadávány do konzole, která 
bude součástí programu.

Konkrétní příkazy, které mají být implementovány jsou:
\begin{itemize}
	\item cp - zkopíruje soubor
	\item mv - přesune soubor
	\item rm - smaže soubor
	\item mkdir - vytvoří adresář
	\item rmdir - smaže adresář
	\item ls - vypíše obsah adresáře
	\item cat - vypíše obsah souboru
	\item cd - změní aktuální cestu do adresáře
	\item pwd - vypíše aktuální cestu
	\item info - vypíše informace o soubrou/ adresáři
	\item incp - zkopíruje soubor pevného disku do pseudo FS
	\item outcp - zkopíruje soubor z pseudo FS na pevný disk
	\item load - načte soubor z pevného disku a vykonavá příkazy z tohoto file
	\item format - formátuje soubor na pseudo FS
	\item slink - vytvoří symbolický link
\end{itemize}
%
%
\chapter{Analýza}
Programu se předá název souboru jako argument při spuštění. Do tohoto souboru je ukládán pseudo FS.
\section{File system}
File systém má 4 hlavní struktury, ze kterých se skládá. Tyto struktury jsou ve FS poskládány dle
obrázku \ref{img:fs_structure}.

Tyto struktury jsou:
\begin{enumerate}
	\item blok (anglicky block) - uchovává data
	\item inode - uchovává metadata o souboru/ adresáři
	\item bitmapa - mapa určující obsazenost bloků/ inode
	\item superblock - uchovává metadata FS a adresy ostatních struktur
\end{enumerate}
%
\begin{figure}[H]
	\centering
	\includegraphics[width=\textwidth]{img/fs_structure.png}
	\caption{Struktura file systému}
	\label{img:fs_structure}
\end{figure}
%
\subsection{Blok}
Je to blok uložiště o konstantní velikosti.
%
\subsection{Inode}
Je to struktura, která reprezentuje každý adresář a soubor ve FS. Uchovává základní informace o
adresáři/ souboru jako například velikost, ID a další. V inode jsou také uchovávány odkazy na 
bloky.

Každý adresář má právě jeden blok a tudíž je limitovaný počet položek, které mohou být v adresáři
uloženy. Soubory mají proměnné počty bloků, ale stále jsou limitovány maximálním počtem bloků.
Maximální počet bloků je také spjatý s velikostí bloku.

Na obrázku \ref{img:inode_struct} je znázorněna struktura inode. V tomto obrázku jsou indirect
bloky až 13 a 14 odkaz inode. V této práci je ale struktura inodu zjednodušená a má pouze 5
přímých odkazů, 1 nepřímý odkaz první úrovně (odkaz 13 v obrázku) a 1 nepřímý odkaz druhé úrovně
(odkaz 14 v obrázku).

\begin{figure}[H]
	\centering
	\def\svgwidth{\columnwidth}
	\input{inode_struct.pdf_tex}
	\caption{Struktura inode}
	Zdroj: https://en.wikipedia.org/wiki/Inode\_pointer\_structure
	\label{img:inode_struct}
\end{figure}
%
\subsection{Bitmapa}
Pro každý blok/ inode je v příslušné bitmapě jeden bit, který určuje zda-li je blok/ inode
na té pozici obsazený.
%
\subsection{Superblock}
Mimo uchovávání metadat, jako například "podpis" tvůrce FS, uchovává umístění ostatních částí.
V superblocku se uchovávají adresy bitmap, inode a bloků. Je zde také zaznamenán počet bloků
a velikost jednoho bloku.
%
\section{Příkazy}
Příkazy mají fungovat na principu stejnojmenných příkazů v systému Linux. Protože příkazy mají být implementovány 
nad zjednodušeným pseudo FS, tak některé příkazy mohou fungovat trochu jinak než v typickém Linuxu nebo
mohou mít úplně jinou funkci (např. příkaz \texttt{info}). Většina příkazů má jeden nebo dva parametry,
které odkazují na soubor nebo adresář ve FS. 
Všechny soubory a adresáře jsou zadány absolutní nebo relativní cestou, proto musí být FS schopný
najít soubor z aktuální pozice i z root adresáře FS.
%
%
%
\chapter{Programátorská dokumentace}
Program je psán v jazyce C, využítím pouze standardních knihoven.
%
\section{Moduly}
Moduly do kterých je aplikce rozdělená.
\begin{itemize}
	\item \texttt{main} - spouštěcí modul programu
	\item \texttt{general\_functions} - poskytuje strukturu linked list a makra
	\item \texttt{console} - modul spouští konzoli, kterou uživatel ovládá program.
		Rozkládá vstup na části a volá příslušné funkce
	\item \texttt{fs\_manager} - zpracuje vstup, který převezme od modulu \texttt{console} a volá
		obslužné funkce z modulu \texttt{commands} se zpracovanými vstupy
	\item \texttt{commands} - vykonává funkce příkazů a provádí změny ve FS
	\item \texttt{file\_system} - poskytuje funkce manipulující s FS a použité FS struktury
\end{itemize}
%
\section{Implementace příkazů}
Všechny příkazy jsou implementovány v modulu \texttt{commands}. Každý příkaz využívá nějákou funkci z
modulu \texttt{file\_system}, aby vykonal akci nad FS.
\subsection{cp}
Prototyp: \texttt{int copy(int32\_t sparent, int32\_t tparent, char* sname, char *tname)}. Vstupy funkce
jsou jméno zdroje (\texttt{sname}) a cíle (\texttt{tname}) a ID adresářú ve kterých se tyto jmena mají nacházet. Funkce 
hledá zdroj a cíl v adresáři podle jména a podle toho, zda úspěšně najde výsledek reaguje. Pokud \texttt{sname}
neexistuje, funkce končí chybou. Na cíl funkce reaguje podle toho, zda-li \texttt{tname} existuje, je adresář nebo
je soubor. Pokud \texttt{tname} neexistuje, vytvoří se v adresáři s ID \texttt{tparent} soubor se jménem \texttt{tname}.
Pokud je \texttt{tname} adresář vytvoří se v něm soubor se jménem \texttt{sname}. Pokud je \texttt{tname}
existující soubor, vymažou se data souboru a nahradí se daty souboru \texttt{sname}.
\subsection{mv}
Prototyp: \texttt{int move(int32\_t sparent, int32\_t tparent, char* sname, char *tname)}. Chová se
podobně jako cp, ale funguje velmi odlišně. Protože není třeba zachovat původní soubor \texttt{sname}, tak příkaz
jednoduše přesune/ přejmenuje adresářovou položku \texttt{sname} do/ na \texttt{tname}.
\subsection{rm}
Prototyp: \texttt{int remove\_file(int32\_t where, char *name)}. Hledá položku se jménem
\texttt{name} v adresáři s inode ID where. Vrací chybu, pokud soubor není nalezen. Pokud je soubor nalezen,
jsou vymazána data souboru a položka souboru z adresáře \texttt{where}.
\subsection{mkdir}
Prototyp: \texttt{int make\_directory(char *name, int32\_t parent\_nid)}. Načte inode s ID \texttt{parent\_nid},
zkontroluje, zda-li jméno name již existuje a pokud neexistuje, tak vytvoří nový inode pro adresář a přidá
položku s \texttt{name} do načteného adresáře.
\subsection{rmdir}
Prototyp: \texttt{int remove\_directory(int32\_t node\_id)}. Načte inode s ID \texttt{node\_id}, zkontroluje,
zda-li je adresář prázdny a následně uvolní blok, do kterého jsou ukládány položky a vynuluje inode s \texttt{node\_id}.
\subsection{ls}
Prototyp: \texttt{int list\_dir\_contents(int32\_t node\_id)}. Načte inode s ID \texttt{node\_id}, pokud se
jedná o adresář, načítá a vypisuje položky v něm uložené.
\subsection{cat}
Prototyp: \texttt{int cat\_file(int32\_t where, char *name)}. Načte inode adresáře \texttt{where} a hledá v něm 
položku \texttt{name}. Po nalezení položky, vypíše její obsah. Načítá po velikosti bloku.
\subsection{cd}
Prototyp: \texttt{int change\_dir(int32\_t target\_id)}. Načte inode s ID \texttt{target\_id} a pokud se jedná
o adresář, nastaví globální proměnou \texttt{position}, která určuje aktuální pozici ve FS, na načtený inode.
\subsection{pwd}
Prototyp: \texttt{int print\_working\_dir()}. Funkce postupně prochází adresáře pomocí ".." (předchozí adresář),
dokud nedojde do root adresáře. Průchodem ukládá každý prošlý adresář do linked listu a ten po dosažení root
adresáře vypíše.
\subsection{info}
Prototyp: \texttt{int node\_info(int32\_t where, char *name)}. Vypíše informace o adresáři nebo souboru 
\texttt{name} nacházející se v adresáři s inode ID \texttt{where}.
\subsection{incp}
Prototyp: \texttt{int in\_copy(char *source, int32\_t t\_node, char *source\_name, char *target\_name)}.
Načte soubor z uložiště mimo FS této práce a zkopíruje data tohoto souboru do pseudo FS. Zdroj souboru
je \texttt{source} a \texttt{source\_name} je pouze název souboru v případě, že soubor byl zadán cestou.
Dále je načten inode s ID \texttt{t\_node}, který by měl adresář, ve kterém se hledá položka
\texttt{target\_name}. Příkaz reaguje stejně jako \textbf{cp} na existenci a typ souboru/ adresáře
\texttt{target\_name}.
\subsection{outcp}
Prototyp: \texttt{int out\_copy(int32\_t where, char *name, char *target)}. Plní opačnou funkci k \textbf{incp}.
Tedy koupíruje soubor se jménem \texttt{name} v adresáře s inode ID \texttt{where} do souboru \texttt{target}, 
který se nachází mimo pseudo FS.
\subsection{load}
Prototyp: \texttt{void load(char *file)}. Tento příkaz není implementován v modulu
\texttt{commands}. Funkce načte soubor \texttt{file} a provádí příkazy, které jsou v souboru uloženy.
Formát souboru je jeden příkaz na řádku. Tato funkce je implementována v modulu \texttt{console}, protože
musí rozložit načtený vstup a volat příslušné obslužné funkce.
\subsection{format}
Prototyp: \texttt{int format(char *size)}. Je druhý příkaz, který není implementovaný v modulu \texttt{commands},
ale je implementován v modulu \texttt{fs\_manager}. Je očekáván argument jako číslo s jednotkami (např. 100MB). 
Po převední jednotek na číslo, se začne vytvářet FS do souboru, který byl zadán jako argument programu. Nejdříve
se vytvoří struktura \texttt{superblock}, do které jsou spočítány adresy počátků ostatních částí FS. Adresy jsou
počítány nejdříve pomocí inkrementu 1:5. Tedy 1B do bitmapy inode (8 inode v každém inkrementu) ku 5B do bitmapy
bloků (tedy 40 bloků). Pokud již není možné do zadané velikosti FS vložit tento inkrement je zbytek místa doplněn bloky
po 8 (tedy 1B v bitmapě bloků). Postupně se potom zapisují data do souboru. Nejdřive se zapíše \texttt{superblock},
poté první byte inode bitmapy, kde se první inode rovnou inicializuje jako zabraný. Po doplnění 0 do začátku bitmapy
bloků, je stejně jako u inode bitmapy první blok inicializovaný a poté doplněn 0 do inode adresy. Poté je incializován
root inode s ID 1 a zapsán do souboru. Tato inode struktura je poté vynulována a dále zapisována do souboru do dosažení
adresy bloků. Na začátku adresy bloků jsou inicializovány adresářové položky "." a ".." root adresáře a zbytek souboru
je doplněn nulami do spočítané velikost.
\subsection{slink}
Prototyp: \texttt{int symbolic\_link(int32\_t src, int32\_t par, char *name)}. Vytvoří nový inode, který ukazuje
na inode ID \texttt{src}, pomocí direct1. Tento inode je uložen pod jménem \texttt{name} do adresáře s inode ID
\texttt{par}. Podle definice se tedy jedná o "smart link", protože link je vázáný na inode ID, tedy funguje
i při přesunutí linku do jiného adresáře nebo přejmenování zdrojového souboru/ adresáře. "Symbolic link" pouze
ukazuje na položku v adresáři a při přesunutí nebo přejmenování zdrojového souboru/ adresáře, je link rozbit.
%
%
\chapter{Uživatelská dokumentace}
%
\section{Přeložení a spuštění programu}
Jedná se o konzolovou aplikaci. Se zdrojovými soubory je přidán i \textbf{makefile}. Požadavky jsou
překladač \textbf{gcc} a příkaz \textbf{make}. Přeložení aplikace je tedy velmi jednoduché. Stačí 
zavolat příkaz \textbf{make} v adresáři s \textbf{makefile}.

Po provední příkazu \textbf{make} je vytvořen spustitelný soubor \textbf{runfs}. Tento soubor 
požaduje právě jeden parametr, který odkazuje na soubor, kde bude vytvořen pseudo FS.
%
\section{Příkazy}
Všechny argumenty příkazů mohou být zadány absolutní nebo relativní cestou.
\begin{itemize}
	\item cp s1 s2 - zkopíruje soubor s1 do/ na s2 (s2 může být adresář i soubor)
	\item mv s1 s2 - přesune soubor s1 do/ na s2 (pokud s2 neexistuje, s1 je přejmenován na s2)
	\item rm s1 - smaže soubor s1
	\item mkdir a1 - vytvoří adresář se jménem a1
	\item rmdir a1 - smaže adresář se jménem a1
	\item ls a1 - vypíše obsah adresáře a1
	\item cat s1 - vypíše obsah souboru s1
	\item cd a1 - změní aktuální cestu do adresáře a1
	\item pwd - vypíše aktuální cestu
	\item info s1/ a1 - vypíše informace o souboru/ adresáři s1/ a1
	\item incp s1 s2 - zkopíruje soubor s1 pevného disku do pseudo FS adresáře/ souboru s2
	\item outcp s1 s2 - zkopíruje soubor s1 z pseudo FS na pevný disk do s2
	\item load s1 - načte soubor s1 z pevného disku a vykonavá příkazy z tohoto file
	\item format xxxYY- formátuje soubor na pseudo FS, xxx je číslo a YY jsou jednotky (př. 100MB)
	\item slink s1 name - vytvoří symbolický link se jménem name ukazující na s1
\end{itemize}

%
%
\chapter{Závěr}
Cílem práce bylo implementovat file system založený na inode a některé základní příkazy 
pro užití a testování tohoto FS. Oproti reálnému FS využívající inode je ale práce zjednodušena.

Práce je stabilní a příkazy z většiny fungují jako příkazy systému Linux. Rozdíl je například u příkazu
info, který má kompletně jinou funkci v Linuxu než v této práci.

Ovšem i tak jsou na práci věci, které by se daly vylepšit. Především výkonostní věci. Příkladem může
být přepisování souboru některým příkazem jako mv/ cp. Ve stávající implementaci, pokud cílový soubor
existuje, tak jsou nejdříve odstraněny všechny reference na bloky dat tohoto souboru a pak jsou znovu
přiděleny nové bloky. Toto je ovšem poměrně neoptimální řešení, protože pokud na příklad bude zrojový
soubor větší než původní, tak není nutné odstranit všechny reference a znovu přidělit potřebný počet 
bloků. Stačilo by pouze přidat počet bloků, aby bylo možné uložit zdrojový soubor a přepsat již
alokované bloky.

\end{document}
